\section{Abstract}
\begin{center}
  \begin{minipage}[center]{0.7\linewidth}
    This thesis presents a comprehensive survey of the mathematical foundations 
    underlying Generative Adversarial Networks (GANs). As deep generative models 
    that have revolutionized artificial intelligence applications, GANs rely on 
    sophisticated mathematical principles that warrant careful examination. 
    We systematically explore three fundamental theoretical frameworks that 
    inform GAN development: game theory, which models the adversarial interplay 
    between generator and discriminator networks; information theory, which 
    provides insights into the information-theoretic limits and efficiency of 
    generative modeling; and optimal transport theory, which offers geometric 
    interpretations of the learning process and convergence properties. This 
    survey synthesizes key theoretical results, examines their interconnections, 
    and discusses their implications for the design, analysis, and improvement 
    of GAN architectures. By elucidating these mathematical underpinnings, 
    this thesis aims to provide researchers and practitioners with a deeper 
    understanding of GAN behavior and limitations.
  \end{minipage}
\end{center}

%%% Local Variables:
%%% mode: latex
%%% TeX-master: "../thesis.tex"
%%% End:
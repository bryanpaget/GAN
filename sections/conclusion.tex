\section{Conclusion}

In Section~\ref{sec:game-theory} we looked at game theory and how the GAN
algorithm can be cast as a game between two competing neural networks. The
minimax strategy for $D$ was $D(x) = {1 \over 2}$ for all $x$, which, if
attained, would make it impossible for the generator to minimize the value
function, since $G$'s actions would no longer affect the actions of $D$. This
strategy makes sense for $D$, since it is the best anyone can do when confronted
with maximum uncertainty.

Section~\ref{sec:information-theory} included two information perspectives on
GAN training. Theorem~\ref{thm:limiting} considered the theoretical, limiting
behaviour of GANs and~\ref{thm:info-objective} considered what happens at each
step of the optimization. This is similar to a macroscopic and microscopic view
of GAN training.

Section~\ref{sec:optimal-transport} introduced optimal transport and showed how
the GAN algorithm has benefited greatly from the application of the
Kantorovich-Rubinstein distance from optimal transport. It takes a great deal of
theoretical understanding to build a well-functioning learning system. By
proposing a variant of the GAN algorithm based on optimal transport
theory,~\cite{ref:arjovsky-2017} have opened up an additional theoretical avenue
of research. Future research for GANs can come from game theory, information
theory and optimal transport.

TODO:Should include a statement of what the thesis contributes to research, what
do you suggest and where does this lead you?  Something about an application?


%%% Local Variables:
%%% mode: latex
%%% TeX-master: "../thesis"
%%% End:

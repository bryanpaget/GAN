\section{Conclusion}

In Section~\ref{sec:game-theory} we looked at game theory and how the GAN
algorithm can be cast as a game between two competing neural networks. The
minimax strategy for $D$ was $D(x) = {1 \over 2}$ for all $x$, which, if
attained, would make it impossible for the generator to minimize the value
function, since $G$'s actions would no longer affect the actions of $D$. This
strategy makes sense for $D$, since it is the best anyone can do when confronted
with maximum uncertainty.

Section~\ref{sec:information-theory} included two information theoretic
perspectives on GAN training. Theorem~\ref{thm:limiting} considered the
theoretical, limiting behaviour of GANs and~\ref{thm:info-objective} considered
what happens at each step of the optimization. This is similar to a macroscopic
and microscopic view of GAN training.

Section~\ref{sec:optimal-transport} introduced optimal transport and showed how
the GAN algorithm has benefited greatly from the application of the
Kantorovich-Rubinstein distance from optimal transport. It takes a great deal of
theoretical understanding to build a well-functioning learning system. By
proposing a variant of the GAN algorithm based on optimal transport
theory,~\cite{ref:arjovsky-2017} have opened up an additional theoretical avenue
of research. Future research for GANs can come from game theory, information
theory and optimal transport.

This thesis provided much of the necessary background material for anyone
wanting to get up to speed on the theory of generative adversarial networks
(GANs). The most important thing going forward with GAN research may be the
detection of fake news related media. GANs have made it incredibly easy to
produce fake images and videos. There are many apps to produce deep fakes, just
use your favorite search engine to find them.

The following quote from \cite{ref:goodfellow-original} hints at the social
impact the GAN algorithm may have in the near future.

\begin{quote}
  \itshape The generative model can be thought of as analogous to a team of
  counterfeiters, trying to produce fake currency and use it without detection,
  while the discriminative model is analogous to the police, trying to detect
  the counterfeit currency. Competition in this game drives both teams to
  improve their methods until the counterfeits are indistinguishable from the
  genuine articles.
\end{quote}

The competition between counterfeiters and police may be soon played out by the
producers of fake images and videos, to be used in fake news, and concerned
researchers. See \forloop{x}{1}{ \value{x}<11 }{ \cite{ref:df\arabic{x}}, }
\cite{ref:df11} and \cite{ref:df12} for more information on fake image and video
detection.

%%% Local Variables:
%%% mode: latex
%%% TeX-master: "../thesis"
%%% End:

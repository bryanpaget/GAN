\section*{Outline of Contributions}

One of the goals I set out for myself when I started this thesis was
to provide a document rich in theory and intuition that would be
useful for anyone wanting to get up to speed on the theory of
GANs. Hence I have broken the theory into three different sections,
\textit{game theory}, \textit{information theory}, and \textit{optimal
  transport theory}.  This document should be equally useful for
someone trying to understand the algorithm in order to implement a
solution to some real world problem. To that end, I have provided an
in-depth introduction to the theory required to understand the GAN
algorithm.

\begin{itemize}
\item Section~\ref{sec:game-theory} contains definitions and examples
  related to the Nash equilibrium, which the original paper mentions
  but does not define.  I have included derivations of the value
  function since the original paper contains no such derivation. See
  sections~\ref{sec:derivation},~\ref{sec:derivation-d},
  and~\ref{sec:derivation-g}.
\item Section~\ref{sec:information-theory} contains relevant
  definitions to information theoretic quantities used all the time in
  machine learning which are rarely defined properly in machine
  learning literature.
\item Section~\ref{sec:optimal-transport} sheds light on what the GAN
  algorithm is really doing, i.e.\ finding an efficient mapping from
  one probability distribution to another. The history of optimal
  transport is presented along with the Wasserstein GAN.
  \item In Section~\ref{sec:optimization-dynamics} for a more rigorous
    proof on training dynamics.  The training dynamics have been cast
    as dynamics between the generator and the discriminator.  Where
    the discriminator is optimized to force the value function into an
    approximation of the Jensen-Shannon divergence and the generator
    is optimized to minimize this divergence, as a moving target.
  \item Section~\ref{sec:info-value-function} contains a unique
    information theoretic theorem about the GAN value function.
\item In Section~\ref{sec:difficulty}, I provide an illustrative
  example demonstrating the difficulty in finding a Nash equilibrium
  is not always easy.
\end{itemize}

%%% Local Variables:
%%% mode: latex
%%% TeX-master: "../thesis.tex"
%%% End:

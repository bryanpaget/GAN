\section{Appendix}
\subsection{Mathematical Proofs and Derivations}
This appendix contains detailed mathematical proofs and derivations supporting
key results presented in the thesis. We include justifications for fundamental
limits, optimization dynamics of GANs, and analysis of minimax formulations.

\subsection{Limit Justification}
\label{sec:limit-justification}
We prove the limit $\lim_{x \to 0} x\log{x} = 0$ using L'Hôpital's rule, which is fundamental for analyzing entropy calculations in information theory.
\begin{align}
	\label{justification:lhospital}
	\lim_{x \to 0} x\log{x} & = \lim_{x \to 0} {\log{x} \over {1 \over x}}        \\
	                        & = \lim_{x \to 0} {{1 \over x} \over {-1 \over x^2}} \\
	                        & = \lim_{x \to 0} {-x^2 \over x}                     \\
	                        & = \lim_{x \to 0} -x                                 \\
	                        & = 0
\end{align}

\subsection{Optimization Dynamics}
\label{sec:proof-for-jsd-thing}
We derive the relationship between the Jensen-Shannon divergence (JSD) and the GAN objective function. This shows that minimizing the GAN loss is equivalent to minimizing the JSD between the real and generated data distributions.
\begin{small}
	\begin{align}
		 & \mathbb{E}\left[\log{\ptx \over {\pgx + \ptx}}\right] + \mathbb{E}\left[\log{\pgx \over {\pgx + \ptx}}\right] + \log 4 - \log 4 \nonumber \\
		 & = \sum_{x} \ptx \log{\ptx \over {\pgx + \ptx}} + \sum_{x} \pgx \log{\pgx \over {\pgx + \ptx}} + \log 4 - \log 4 \nonumber                 \\
		 & = \sum_{x} \ptx \log{\ptx \over {\pgx + \ptx}} + \sum_{x} \pgx \log{\pgx \over {\pgx + \ptx}} + \log 2 + \log 2 - \log 4 \nonumber        \\
		 & = \sum_{x} \ptx \log{\ptx \over {\pgx + \ptx}} + \sum_{x} \pgx \log{\pgx \over {\pgx + \ptx}} \nonumber                                   \\
		 & \quad + \sum_{x} \ptx \log 2 + \sum_{x} \pgx \log 2 - \log 4 \nonumber                                                                    \\
		 & = \sum_{x} \ptx \log{2 \ptx \over {\pgx + \ptx}} + \sum_{x} \pgx \log{2 \pgx \over {\pgx + \ptx}} - \log 4 \nonumber                      \\
		 & = \sum_{x} \ptx \log{\ptx \over {(\pgx + \ptx) \over 2}} + \sum_{x} \pgx \log{\pgx \over {(\pgx + \ptx) \over 2}} - \log 4 \nonumber      \\
		 & = \KL{\ptx}{(\pgx + \ptx) \over 2} + \KL{\pgx}{(\pgx + \ptx) \over 2} - \log 4 \nonumber                                                  \\
		 & = 2 \cdot \JSD{\pgx}{\ptx} - \log 4 \label{eq:jsd-derivation}
	\end{align}
\end{small}
Equation \ref{eq:jsd-derivation} establishes the connection between the GAN
objective and JSD, demonstrating that the optimal discriminator corresponds to
the JSD between the real and generated distributions.

\subsection{Minimax Analysis of GAN Objectives}
\label{sec:minimax-analysis}
We analyze the minimax and maximin values of the discriminator ($\VD$) and generator ($\VG$) objectives, providing theoretical foundations for understanding GAN convergence properties.

\subsubsection{Minimax of $\VD$}
\label{sec:minimax-vd}
\begin{align}
	\overline{\VD} & = \min_{\phi}\max_{\theta} V_{\D}(\theta, \phi) \nonumber                                                                                                     \\
	               & = \min_{\phi}\max_{\theta} \left( \mathbb{E}\left[\log{\D(x)}\right] + \mathbb{E}\left[\log(1 - \D(\G(z)))\right] \right) \nonumber                           \\
	               & = \min_{\phi}\left(\max_{\theta} \left( \mathbb{E}\left[\log{\D(x)}\right] + \mathbb{E}\left[\log(1 - \D(\G(z)))\right] \right) \right) \nonumber             \\
	               & = \min_{\phi}\left(\mathbb{E}\left[\log{\pt \over {\pg + \pt}}\right] + \mathbb{E}\left[\log\left(1 - {\pt \over {\pg + \pt}}\right)\right] \right) \nonumber \\
	               & = \mathbb{E}\left[\log{\pt \over {\pt + \pt}}\right] + \mathbb{E}\left[\log\left(1 - {\pt \over {\pt + \pt}}\right)\right] \nonumber                          \\
	               & = \mathbb{E}\left[\log{1 \over 2}\right] + \mathbb{E}\left[\log{\left(1 - {1 \over 2}\right)}\right] \nonumber                                                \\
	               & = \log{1 \over 2} + \log{1 \over 2} \nonumber                                                                                                                 \\
	               & = \log{1 \over 4} \label{eq:minimax-vd}
\end{align}
Equation \ref{eq:minimax-vd} shows that the minimax value of the discriminator
objective is $\log(1/4)$, achieved when the generator perfectly matches the
real data distribution.

\subsubsection{Maximin of $\VD$}
\label{sec:maximin-vd}
\begin{align}
	\underline{\VD} & = \max_{\theta}\min_{\phi} V_{\D}(\theta, \phi) \nonumber                                                                                         \\
	                & = \max_{\theta}\min_{\phi} \left( \mathbb{E}\left[\log{\D(x)}\right] + \mathbb{E}\left[\log(1 - \D(\G(z)))\right] \right) \nonumber               \\
	                & = \max_{\theta}\left(\min_{\phi} \left( \mathbb{E}\left[\log{\D(x)}\right] + \mathbb{E}\left[\log(1 - \D(\G(z)))\right] \right) \right) \nonumber \\
	                & = \max_{\theta}\left(\left( \mathbb{E}\left[\log{\D(x)}\right] + \mathbb{E}\left[\log(1 - \D(x))\right] \right) \right) \nonumber                 \\
	                & = \max_{\theta}\left(\left( \mathbb{E}\left[\log{\D(x)} + \log(1 - \D(x))\right] \right) \right) \nonumber                                        \\
	                & = \mathbb{E}\left[\log{\pt \over {\pt + \pt}}\right] + \mathbb{E}\left[\log\left(1 - {\pt \over {\pt + \pt}}\right)\right] \nonumber              \\
	                & = \mathbb{E}\left[\log{1 \over 2}\right] + \mathbb{E}\left[\log{\left(1 - {1 \over 2}\right)}\right] \nonumber                                    \\
	                & = \log{1 \over 2} + \log{1 \over 2} \nonumber                                                                                                     \\
	                & = \log{1 \over 4} \label{eq:maximin-vd}
\end{align}
Equation \ref{eq:maximin-vd} demonstrates that the maximin value of the
discriminator objective also equals $\log(1/4)$, confirming the existence of a
saddle point in the GAN game.

\subsubsection{Minimax of $\VG$}
\label{sec:minimax-vg}
\begin{align}
	\overline{\VG} & = \min_{\theta}\max_{\phi} \VG \nonumber                                                                                                           \\
	               & = \min_{\theta}\max_{\phi} \left( -\mathbb{E}\left[\log{\D(x)}\right] - \mathbb{E}\left[\log(1 - \D(\G(z)))\right] \right) \nonumber               \\
	               & = \min_{\theta}\left(\max_{\phi} \left( -\mathbb{E}\left[\log{\D(x)}\right] - \mathbb{E}\left[\log(1 - \D(\G(z)))\right] \right) \right) \nonumber \\
	               & = \min_{\theta}\left( -\mathbb{E}\left[\log{\D(x)}\right] - \mathbb{E}\left[\log(1 - \D(x))\right] \right) \nonumber                               \\
	               & = \min_{\theta}\left( -\mathbb{E}\left[\log{\D(x)} + \log(1 - \D(x))\right] \right) \nonumber                                                      \\
	               & = -\mathbb{E}\left[\log{\pt \over {\pt + \pt}} + \log\left(1 - {\pt \over {\pt + \pt}}\right) \right] \nonumber                                    \\
	               & = - \mathbb{E}\left[\log{1 \over 2} + \log{\left(1 - {1 \over 2}\right)}\right] \nonumber                                                          \\
	               & = - \log{1 \over 2} - \log{1 \over 2} \nonumber                                                                                                    \\
	               & = - \log{1 \over 4} \label{eq:minimax-vg}
\end{align}
Equation \ref{eq:minimax-vg} shows that the minimax value of the generator
objective is $-\log(1/4)$, which is the negative of the discriminator's minimax
value.

\subsubsection{Maximin of $\VG$}
\label{sec:maximin-vg}
\begin{align}
	\underline{\VG} & = \max_{\phi} \min_{\theta} \VG \nonumber                                                                                                          \\
	                & = \max_{\phi}\min_{\theta} \VG \nonumber                                                                                                           \\
	                & = \max_{\phi}\min_{\theta} \left( -\mathbb{E}\left[\log{\D(x)}\right] - \mathbb{E}\left[\log(1 - \D(\G(z)))\right] \right) \nonumber               \\
	                & = \max_{\phi}\left(\min_{\theta} \left( -\mathbb{E}\left[\log{\D(x)}\right] - \mathbb{E}\left[\log(1 - \D(\G(z)))\right] \right) \right) \nonumber \\
	                & = \max_{\phi}\left( -\mathbb{E}\left[\log{\pt \over {\pt + \pg}} + \log(1 - {\pt \over {\pt + \pg}}) \right] \right) \nonumber                     \\
	                & = -\mathbb{E}\left[\log{\pt \over {\pt + \pt}} + \log\left(1 - {\pt \over {\pt + \pt}}) \right] \nonumber                                          \\
	                & = - \mathbb{E}\left[\log{1 \over 2} + \log{\left(1 - {1 \over 2}\right)}\right] \nonumber                                                          \\
	                & = - \log{1 \over 2} - \log{1 \over 2} \nonumber                                                                                                    \\
	                & = - \log{1 \over 4} \label{eq:maximin-vg}
\end{align}
Equation~\ref{eq:maximin-vg} confirms that the maximin value of the generator
objective equals $-\log(1/4)$, consistent with the minimax value and
demonstrating the stability of the GAN optimization problem.

%%% Local Variables:
%%% mode: latex
%%% TeX-master: "../thesis"
%%% End:

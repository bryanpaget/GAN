%===============================================================================
% PREAMBLE
%===============================================================================
% Document Class and Encoding (for pdfLaTeX)
\usepackage[utf8]{inputenc}
\usepackage[T1]{fontenc}
\usepackage{lmodern}
\usepackage{textcomp}
%===============================================================================
% PAGE LAYOUT AND FORMATTING
%===============================================================================
% Geometry and Margins
\usepackage{geometry}
\geometry{a4paper, margin=1.25in, top=1.25in, bottom=1.25in}
% Line Spacing
\usepackage{setspace}
\setstretch{1.4}
% Font Selection (using Latin Modern instead of Times New Roman for better compatibility)
% \usepackage{yfonts}  % For old-style fonts - comment out if not needed

% Bibliography setup
\usepackage[backend=biber,style=authoryear,citestyle=authoryear]{biblatex}
\addbibresource{thesis.bib}

% Other packages
\usepackage{amsmath,amssymb,amsthm}
\usepackage{graphicx}
\usepackage{hyperref}
% Section Formatting
\usepackage{titlesec}
\titleformat{\section}[display]
  {\large\filcenter\scshape}
  {\titlerule[0.01pc]\vspace{0.15cm}\large Section \thesection}
  {0pc}
  {\vspace{0.15cm}\titlerule[0.01pc]\vspace{0.15cm}\LARGE}
% Table of Contents Settings
\renewcommand{\contentsname}{Table of Contents}
\setcounter{tocdepth}{2}
% Footnote Formatting
\renewcommand{\thefootnote}{\arabic{footnote}}
% Figure and Table Formatting
\usepackage[font={small,it}]{caption}
\usepackage{subcaption}
\usepackage{float}
\usepackage{booktabs}  % Professional quality tables
\usepackage{diagbox}   % Diagonal lines in tables
\usepackage{wrapfig}   % Wrap text around figures
%===============================================================================
% MATHEMATICAL PACKAGES AND SETTINGS
%===============================================================================
% Core Math Packages
\usepackage{amsmath}
\usepackage{amssymb}
\usepackage{amsthm}
% Math Numbering
\numberwithin{equation}{section}
% Theorem Environments
\theoremstyle{plain}
\newtheorem{theorem}{Theorem}[section]
\newtheorem{claim}[theorem]{Claim}
\newtheorem{definition}[theorem]{Definition}
\newtheorem{proposition}[theorem]{Proposition}
\newtheorem{lemma}[theorem]{Lemma}
\newtheorem{corollary}[theorem]{Corollary}
\newtheorem{conjecture}[theorem]{Conjecture}
\newtheorem{problem}[theorem]{Problem}
% Unnumbered Theorem Environments
\newtheorem*{summary}{Summary}
\newtheorem*{observation}{Observation}
\newtheorem*{example}{Example}
\newtheorem*{remark}{Remark}
% Math Operators
\DeclareMathOperator*{\argmax}{arg\,max}
\DeclareMathOperator*{\argmin}{arg\,min}
%===============================================================================
% GRAPHICS AND VISUALIZATION
%===============================================================================
% Graphics and TikZ
\usepackage{graphicx}
\graphicspath{{./images/}{./figures/}}
\usepackage{tikz}
\usepackage{tikz-cd}
\usepackage{pgfplots}
% PGFPlots Configuration with Safeguards
\pgfplotsset{
  width=7cm,
  compat=1.18,
  % Handle unbounded coordinates gracefully
  unbounded coords=discard,
  clip mode=individual,
  clip=true,
  % Set default axis limits to prevent issues
  enlargelimits=false,
  axis equal image=false
}
% TikZ Libraries
\usetikzlibrary{
  shapes, arrows, matrix, chains,
  positioning, calc, decorations.pathreplacing,
  patterns, decorations.markings
}
% TikZ Styles
\tikzstyle{mystar} = [star, star points=6, draw]
\tikzstyle{node1} = [circle, draw, node distance=2cm]
\tikzstyle{node2} = [circle, draw, node distance=2cm]
\tikzstyle{node3} = [circle, draw, node distance=1cm]
\tikzstyle{t} = [node distance=1.4cm]
\tikzstyle{decision} = [diamond, draw, node distance=3cm]
\tikzstyle{nn3} = [circle, draw, node distance=3cm]
\tikzstyle{nn2} = [circle, draw, node distance=2cm]
\tikzstyle{nn} = [circle, draw, node distance=1cm]
\tikzstyle{true} = [rectangle, draw, rounded corners, node distance=3cm]
\tikzstyle{noise} = [rectangle, draw, rounded corners, node distance=3cm]
\tikzstyle{brain} = [star, star points=10, draw, rounded corners, node distance=3cm]
\tikzstyle{line} = [draw, -latex']
\tikzstyle{cloud} = [draw, ellipse, node distance=3cm]
\tikzstyle{pic} = [node distance=3cm]
% PGF Plots Functions
\pgfmathdeclarefunction{gauss}{2}{
  \pgfmathparse{1/(#2*sqrt(2*pi))*exp(-((x-#1)^2)/(2*#2^2))}
}
% Safe TikZ Environments
\newenvironment{safetikz}[1][]{
  \begin{tikzpicture}
    \pgfplotsset{
      unbounded coords=discard,
      clip=true,
      #1
    }
}{
  \end{tikzpicture}
}
\newenvironment{safeaxis}[1][]{
  \begin{axis}[
    width=7cm,
    height=5cm,
    xmin=0, xmax=100,
    ymin=-10, ymax=10,
    unbounded coords=discard,
    clip=true,
    grid=major,
    #1
  ]
}{
  \end{axis}
}
% Safe Plot Commands
\newcommand{\safeplot}[3][]{
  \addplot[#1] coordinates {
    #2
  };
  \addlegendentry{#3}
}
\newcommand{\safecoord}[2]{
  \pgfmathparse{#2}
  \ifnum\pgfmathresult>-10000
    \ifnum\pgfmathresult<10000
      (#1,#2)
    \fi
  \fi
}
% Rotating Figures
\usepackage{rotating}
%===============================================================================
% ALGORITHMS AND CODE LISTINGS
%===============================================================================
% Algorithm Package
\usepackage[linesnumbered, ruled]{algorithm2e}
% Code Listings
\usepackage{listings}
\usepackage{xcolor}
\lstset{
  language=Python,
  showstringspaces=false,
  basicstyle={\small\ttfamily},
  numbers=none,
  tabsize=4,
  keywordstyle=\color{blue},
  commentstyle=\color{green!50!black},
  stringstyle=\color{red},
  frame=single,
  frameround=tttt,
  backgroundcolor=\color{gray!10}
}
%===============================================================================
% HYPERLINKS AND PDF SETTINGS
%===============================================================================
% Hyperref Package
\usepackage[pdftex]{hyperref}
\hypersetup{
  pdftitle={Generative Adversarial Networks: An Overview},
  pdfauthor={Bryan Paget},
  pdfkeywords={Machine Learning, Deep Learning, GANs, Statistics, Game Theory},
  pdfsubject={Statistics, Game Theory, Machine Learning},
  pdffitwindow=true,
  colorlinks=true,
  linkcolor=blue,
  citecolor=green,
  filecolor=magenta,
  urlcolor=orange,
  pdfstartpage=2
}
% Cross-Referencing
\usepackage{nameref}
%===============================================================================
% UTILITY PACKAGES AND COMMANDS
%===============================================================================
% Page Numbering and References
\usepackage{lastpage}  % For referencing the last page
\usepackage{afterpage}  % For creating blank pages
\usepackage{forloop}    % For loop constructs
% Enumerate Package
\usepackage{enumerate}
% Blank Page Command
\newcommand{\blankpage}{
  \null
  \thispagestyle{empty}
  \newpage
}
%===============================================================================
% CUSTOM MATHEMATICAL COMMANDS
%===============================================================================
% Value Functions
\newcommand{\V}{V(\phi, \theta)}
\newcommand{\Va}{\mathcal{V}(\phi, \theta)}
\newcommand{\VD}{V_{D_\theta}(\phi, \theta)}
\newcommand{\VG}{V_{G_\phi}(\phi, \theta)}
% Spaces and Distributions
\newcommand{\target}{\mathcal{X}}
\newcommand{\prior}{\mathcal{Z}}
% Generator Commands
\newcommand{\G}{G_\phi}
\newcommand{\Gt}{G_{\phi_{t}}}
\newcommand{\Gtt}{G_{\phi_{t+1}}}
\newcommand{\Gl}{G_{\infty}}
\newcommand{\Gz}{G_\phi(z)}
\newcommand{\Gzi}{G_\phi(z_i)}
% Discriminator Commands
\newcommand{\D}{D_\theta}
\newcommand{\Dt}{D_{\theta_{t}}}
\newcommand{\Dtt}{D_{\theta_{t+1}}}
\newcommand{\Dl}{D_{\infty}}
% Probability Distributions
\newcommand{\pg}{p_{\phi}}
\newcommand{\pgx}{p_{\phi}(\tilde{x})}
\newcommand{\pgu}{p_{\phi}(u)}
\newcommand{\pgt}{p_{\phi_t}(\tilde{x})}
\newcommand{\pgtu}{p_{\phi_t}(u)}
\newcommand{\pd}{D_\theta(x)}
\newcommand{\pt}{p^*}
\newcommand{\ptx}{p^*(x)}
\newcommand{\ptu}{p^*(u)}
\newcommand{\pz}{p_{\mathcal{Z}}}
\newcommand{\pzz}{p_{\mathcal{Z}}(z)}
\newcommand{\pdg}{D_{\theta}(G_{\phi}(z))}
% General Math Commands
\def\*#1{\mathbf{#1}}
\def\&#1{\mathcal{#1}}
\def\i#1{\textit{#1}}
\def\R{\mathbb{R}}
% Restriction Command
\newcommand{\restr}[2]{\left. #1 \vphantom{\big|} \right|_{#2}}
% Divergence Commands
\newcommand{\Div}[1]{\mathcal{D}_{\text{\tiny#1}}}
\newcommand{\Divgen}[2]{\Div{} \kern-\nulldelimiterspace \left( #1 \kern 0.1em \middle| \middle| \kern 0.1em #2 \right)}
\newcommand{\KL}[2]{\Div{KL} \kern-\nulldelimiterspace \left( #1 \kern 0.1em \middle| \middle| \kern 0.1em #2 \right)}
\newcommand{\JSD}[2]{\Div{JS} \kern-\nulldelimiterspace \left(#1 \kern 0.1em \middle|\middle| \kern 0.1em #2\right)}
\newcommand{\Dstar}[1]{{p^*(#1) \over p^*(#1) + p_{\phi}(#1)}}
\newcommand{\maxV}[1]{\mathbb{E} \left[\log{{\pt \over \pg + \pt}}\right] + \mathbb{E}\left[\log{\pg \over \pg + \pt} \right]}
% Expectation Commands
\newcommand{\E}[2]{\mathbb{E}_{#1}\left[ #2 \right]}
\newcommand{\KLE}[3]{\E{#1\sim #2}{\log{#2 \over #3}}}
\newcommand{\KLd}[2]{\sum_{\x \sim #1} #1 \log{#1 \over #2}}
\newcommand{\KLc}[3]{\int_{#1} #2 \log{#2 \over #3}d#1}
%===============================================================================
% DOCUMENT SETTINGS
%===============================================================================
% Section Counter
\setcounter{section}{0}
%===============================================================================
% Local Variables:
%%% mode: latex
%%% TeX-master: "thesis"
%%% End:

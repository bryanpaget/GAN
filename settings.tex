\usepackage{lastpage} % What does this do?
\usepackage{wrapfig, subcaption, setspace, booktabs}

% Initial letter
\usepackage{lettrine}
\usepackage{Royal}

% Box [thing \ other thing]
\usepackage{diagbox}

% New Page
\usepackage{afterpage}
\newcommand\blankpage{
  \null
  \thispagestyle{empty}
  \addtocounter{page}{-1}
  \newpage
}

% TOC depth
\setcounter{tocdepth}{2}
\usepackage{titlesec}
\usepackage{nameref}% http://ctan.org/pkg/nameref

\titleformat{\section}[display]{\large\filcenter\scshape}{%
  \titlerule[0.01pc]%
  \vspace{0.15cm}%
  \large
  Section \thesection}{0pc}{
  \vspace{0.15cm}%
  \titlerule[0.01pc]%
  \vspace{0.15cm}%
  \LARGE}

%%% Math Packages:
\usepackage{amsmath}
\usepackage{amssymb}
\numberwithin{equation}{section}

%%% Fonts and typography stuff
\usepackage[utf8]{inputenc}
\usepackage[T1]{fontenc}
\usepackage{yfonts}

%%% Footnotes:
\renewcommand{\thefootnote}{\arabic{footnote}}

%%% Figure Font Sizes:
\usepackage[font={small,it}]{caption}
\usepackage{float}

%%% Graphics:
\usepackage{rotating}
\usepackage{tikz}
\usetikzlibrary{shapes, arrows, matrix, chains}
\usepackage{graphicx}
\graphicspath{ {./images/} }
\usepackage{pgfplots}
\pgfplotsset{width=7cm,compat=1.8}
\pgfmathdeclarefunction{gauss}{2}{
  \pgfmathparse{1/(#2*sqrt(2*pi))*exp(-((x-#1)^2)/(2*#2^2))}
}

%%% Block styles
\tikzstyle{mystar} = [star,star points=6, draw]
\tikzstyle{node1} = [circle, draw, node distance=2cm]
\tikzstyle{node2} = [circle, draw, node distance=2cm]
\tikzstyle{node3} = [circle, draw, node distance=1cm]
\tikzstyle{t} = [node distance=1.4cm]
\tikzstyle{decision} = [diamond, draw, node distance=3cm]
\tikzstyle{nn3} = [circle, draw, node distance=3cm]
\tikzstyle{nn2} = [circle, draw, node distance=2cm]
\tikzstyle{nn} = [circle, draw, node distance=1cm]
\tikzstyle{true} = [rectangle, draw, rounded
corners, node distance=3cm]
\tikzstyle{noise} = [rectangle, draw, rounded
corners, node distance=3cm]
\tikzstyle{brain} = [star,star points=10, draw, rounded
corners, node distance=3cm]
\tikzstyle{line} = [draw, -latex']
\tikzstyle{cloud} = [draw, ellipse, node distance=3cm]
\tikzstyle{pic} = [node distance=3cm]

%%% Theorem:
\usepackage{amsthm}
\theoremstyle{plain}
\newtheorem{theorem}{Theorem}[section]
\newtheorem{claim}[theorem]{Claim}
\newtheorem{definition}[theorem]{Definition}
\newtheorem{proposition}[theorem]{Proposition}
\newtheorem{lemma}[theorem]{Lemma}
\newtheorem*{summary}{Summary}
\newtheorem{corollary}[theorem]{Corollary}
\newtheorem{conjecture}[theorem]{Conjecture}
\newtheorem*{observation}{Observation}
\newtheorem*{example}{Example}
\newtheorem*{remark}{Remark}
\newtheorem{problem}[theorem]{Problem}

%%% Geometry:
\usepackage{geometry}

%%% Algorithm:
\usepackage[linesnumbered, ruled]{algorithm2e}

%%% Spacing:
\usepackage{setspace}
\renewcommand{\baselinestretch}{1.4}

%%% Commands:
\newcommand \V {
  V(\phi, \theta)
}

\newcommand \Va {
  \mathcal{V}(\phi, \theta)
}

\newcommand \VD {
  V_{D_\theta}(\phi, \theta)
}

\newcommand \VG {
  V_{G_\phi}(\phi, \theta)
}

% Spaces
\newcommand{\target}{\&X}
\newcommand{\prior}{\&Z}

% Generator
\newcommand{\G}{G_\phi}
\newcommand{\Gt}{G_{\phi_{t}}}
\newcommand{\Gtt}{G_{\phi_{t+1}}}
\newcommand{\Gl}{G_{{\infty}}}
\newcommand{\Gz}{G_\phi(\mathbf{z})}
\newcommand{\Gzi}{G_\phi(\mathbf{z}_i)}

\newcommand{\D}{D_\theta}
\newcommand{\Dt}{D_{\theta_{t}}}
\newcommand{\Dtt}{D_{\theta_{t+1}}}
\newcommand{\Dl}{D_{\infty}}

\newcommand{\pg}{p_{\phi}(\tilde{\mathbf{x}})}
\newcommand{\pgu}{p_{\phi}(\mathbf{u})}
\newcommand{\pgt}{p_{\phi_t}(\tilde{\mathbf{x}})}
\newcommand{\pgtu}{p_{\phi_t}(\mathbf{u})}

\newcommand{\pd}{D_\theta(\mathbf{x})}

\newcommand{\pt}{p^*(\mathbf{x})}
\newcommand{\ptu}{p^*(\mathbf{u})}
\newcommand{\pz}{p_\prior(\mathbf{z})}

\newcommand \pdg {
  D_{\theta}(G_{\phi}(\textbf{z}))
}

\def\*#1{\mathbf{#1}}

\def\&#1{\mathcal{#1}}

\def\i#1{\textit{#1}}

\def\R{\mathbb{R}}

\DeclareMathOperator*{\argmax}{arg\,max}

\DeclareMathOperator*{\argmin}{arg\,min}

\newcommand \restr[2] {
    \left. \kern- \nulldelimiterspace #1 % the function
      \vphantom{\big|} % pretend it's a little taller at normal size
    \right|_{#2} % this is the delimiter
}

\newcommand \Div[1] {
  \mathcal{D}_{\text{\tiny#1}}
}

\newcommand \KL[2] {
  \Div{KL}\left( #1 \middle| \middle| #2 \right)
}

\newcommand \JSD[2] {
  \Div{JS}\left(#1\middle|\middle|#2\right)
}

\newcommand \Dstar[1] {
  {p^*(\mathbf{#1}) \over p^*(\mathbf{#1}) + p_{\phi}(\mathbf{#1})}
}

\newcommand \maxV[1] {
  \mathbb{E} \left[\log{{\pt \over \pg + \pt}}\right] +
  \mathbb{E}\left[\log{\pg \over \pg + \pt} \right]
}

\newcommand \E[2] {
  \mathbb{E}_{#1}\left[ #2 \right]
}

\newcommand \KLE[3] {
  \E{#1\sim #2}{\log{#2 \over #3}}
}

\newcommand \KLd[2] {
  \sum_{\x \sim #1} #1 \log{#1 \over #2}
}

\newcommand \KLc[3] {
  \int_{#1} #2 \log{#2 \over #3}d#1
}

%%% Code
\usepackage{listings}
\usepackage{color}
\lstset{
  language=Python,
  showstringspaces=false,
  basicstyle={\small\ttfamily},
  numbers=none,
  tabsize=4
}

%%% PDF Settings
\usepackage[pdftex]{hyperref}
\hypersetup{
  pdftitle={Generative Adversarial Networks: An Overview},
  pdfauthor={Bryan Paget},
  pdfkeywords={Machine Learning, Deep Learning, GANs, Statistics, Game Theory},
  pdfsubject={Statistics, Game Theory, Machine Learning},
  pdffitwindow=true,
  colorlinks=false,
  linkcolor=orange,
  citecolor=orange,
  filecolor=magenta,
  urlcolor=black,
  pdfstartpage=2
}

%%% Section Counter
\setcounter{section}{0}

\usepackage{enumerate}

%%% Local Variables:
%%% mode: latex
%%% TeX-master: "thesis"
%%% End:
